\appendixchapter{Eventlog File Format}
\label{cha:eventlog-file-format}

This appendix documents the format of the eventlog file. Eventlog
files are written by the simulation (when enabled). Everything
that happens during the simulation is recorded into the file,
  \footnote{With certain granularity of course, and subject to
  filters that were active during simulation}
so the file can later be used to reproduce the history of the
simulation on a sequence chart, or in some other form.

The file is a line-oriented text file. Blank lines and lines beginning
with "\#" (comments) will be ignored. Other lines begin with an
\textit{entry identifier} like \ttt{E} for \textit{Event} or
\ttt{BS} for \textit{BeginSend}, followed by \textit{attribute-identifier}
and \textit{value} pairs. One exception is debug output
(recorded from \ttt{EV<<...} statements), which are represented
by lines that begin with a hyphen, and continue with the actual text.

The grammar of the eventlog file is the following:

\begin{verbatim}
<file> ::= <line>*
<line> ::= <empty-line> | <user-log-message> | <event-log-entry>
<empty-line> ::= CR LF
<user-log-message> ::= - SPACE <text> CR LF
<event-log-entry> ::= <event-log-entry-type> SPACE <parameters> CR LF
<event-log-entry-type> ::= SB | SE | BU | MB | ME | MC | MD | MR | GC | GD |
                           CC | CD | CS | MS | CE | BS | ES | SD | SH | DM | E
<parameters> ::= (<parameter>)*
<parameter> ::= <name> SPACE <value>
<name> ::= <text>
<value> ::= <boolean> | <integer> | <text> | <quoted-text>
\end{verbatim}

The eventlog file must also fulfill the following requirements:
\begin{itemize}
   \item simulation events are in increasing event number and simulation time order
%%   \item FIXME anything more?
\end{itemize}

Here is a fragment of an existing eventlog file as an example:

\begin{filelisting}
E # 14 t 1.018454036455 m 8 ce 9 msg 6
BS id 6 tid 6 c cMessage n send/endTx pe 14
ES t 4.840247053855
MS id 8 d t=TRANSMIT,,#808000;i=device/pc_s
MS id 8 d t=,,#808000;i=device/pc_s

E # 15 t 1.025727827674 m 2 ce 13 msg 25
- another frame arrived while receiving -- collision!
CE id 0 pe 12
BS id 0 tid 0 c cMessage n end-reception pe 15
ES t 1.12489449434
BU id 2 txt "Collision! (3 frames)"
DM id 25 pe 15
\end{filelisting}

\section{Supported Entry Types and Their Attributes}
\label{sec:eventlog-file-format:entry-types}

The following entries and attributes are supported in the eventlog file:

%%
%% generated with tools/stripeventlog.pl from src/eventlog/eventlogentries.txt
%%

% @BEGINFILE tools/eventlog.txt
\tbf{SB} \textit{(SimulationBegin)}: mandatory first line of the eventlog file, followed by an empty line

\begin{itemize}
  \item \tbf{ov} (\textit{omnetppVersion}, int): OMNeT++ version, e.g. 0x0401 (=1025) is release 4.1
  \item \tbf{ev} (\textit{eventlogVersion}, int): eventlog version number
  \item \tbf{rid} (\textit{runId}, string): identifies the simulation run
\end{itemize}

\tbf{SE} \textit{(SimulationEnd)}: optional last non-empty line of the eventlog file, followed by an empty line

\begin{itemize}
  \item \tbf{e} (\textit{isError}, bool): specifies if the simulation terminated due to an error
  \item \tbf{c} (\textit{resultCode}, int): the error code in case of an error, otherwise the normal result code
  \item \tbf{m} (\textit{message}, string): human readable description
\end{itemize}

\tbf{E} \textit{(Event)}: an event that is processing a message, terminated by an empty line

\begin{itemize}
  \item \tbf{\#} (\textit{eventNumber}, eventnumber\_t): unique event number
  \item \tbf{t} (\textit{simulationTime}, simtime\_t): simulation time when the event occurred
  \item \tbf{m} (\textit{moduleId}, int): id of the processing module
  \item \tbf{ce} (\textit{causeEventNumber}, eventnumber\_t): event number from which the message being processed was sent, or -1 if the message was sent from initialize
  \item \tbf{msg} (\textit{messageId}, msgid\_t): id of the message being processed
  \item \tbf{f} (\textit{fingerprints}, string): current simulation fingerprints
\end{itemize}

\tbf{S} \textit{(Snapshot)}: a snapshot of the current simulation state, followed by state entries, and terminated by an emtpy line

\begin{itemize}
  \item \tbf{f} (\textit{fileOffset}, int64\_t): file offset of this entry
  \item \tbf{\#} (\textit{eventNumber}, eventnumber\_t): event number of the last processed event
  \item \tbf{t} (\textit{simulationTime}, simtime\_t): simulation time of the last processed event
\end{itemize}

\tbf{I} \textit{(Index)}: incremental snapshot specifying additional and removed entries with an event number and a line index, followed by an empty line

\begin{itemize}
  \item \tbf{f} (\textit{fileOffset}, int64\_t): file offset of this entry
  \item \tbf{i} (\textit{previousIndexFileOffset}, int64\_t): file offset of the previous index entry
  \item \tbf{s} (\textit{previousSnapshotFileOffset}, int64\_t): file offset of the previous snapshot entry
  \item \tbf{\#} (\textit{eventNumber}, eventnumber\_t): event number of the last processed event
  \item \tbf{t} (\textit{simulationTime}, simtime\_t): simulation time of the last processed event
\end{itemize}

\tbf{abstract} \textit{(Reference)}: base class for index entry references

\begin{itemize}
  \item \tbf{\#} (\textit{eventNumber}, eventnumber\_t): event number of the last referred event
  \item \tbf{b} (\textit{beginEntryIndex}, int): begin index of the referred entry within the event
  \item \tbf{e} (\textit{endEntryIndex}, int): end index of the referred entry within the event
\end{itemize}

\tbf{RF} \textit{(ReferenceFound)}: specifies an eventlog entry found in the snapshot

\begin{itemize}
  \item no parameters
\end{itemize}

\tbf{RA} \textit{(ReferenceAdded)}: specifies an eventlog entry added to the index

\begin{itemize}
  \item no parameters
\end{itemize}

\tbf{RR} \textit{(ReferenceRemoved)}: specifies an eventlog entry removed from the index

\begin{itemize}
  \item no parameters
\end{itemize}

\tbf{abstract} \textit{(ModuleReference)}: base class for entries referring to a module

\begin{itemize}
  \item \tbf{id} (\textit{moduleId}, int): id of the module
\end{itemize}

\tbf{abstract} \textit{(GateReference)}: base class for entries referring to a gate

\begin{itemize}
  \item \tbf{m} (\textit{moduleId}, int): id of module where the gate is
  \item \tbf{g} (\textit{gateId}, int): id of the gate
\end{itemize}

\tbf{abstract} \textit{(ConnectionReference)}: base class for entries referring to a connection

\begin{itemize}
  \item \tbf{sm} (\textit{sourceModuleId}, int): id of the source module identifying the connection
  \item \tbf{sg} (\textit{sourceGateId}, int): id of the gate at the source module identifying the connection
\end{itemize}

\tbf{abstract} \textit{(MessageReference)}: base class for entries referring to a message

\begin{itemize}
  \item \tbf{id} (\textit{messageId}, msgid\_t): id of the message
\end{itemize}

\tbf{abstract} \textit{(ModuleDescription)}: base class for entries describing a module

\begin{itemize}
  \item \tbf{c} (\textit{moduleClassName}, string): C++ class name of the module
  \item \tbf{t} (\textit{nedTypeName}, string): fully qualified NED type name
  \item \tbf{pid} (\textit{parentModuleId}, int): id of the parent module
  \item \tbf{n} (\textit{fullName}, string): full dotted hierarchical module name
  \item \tbf{cm} (\textit{compoundModule}, bool): whether module is a simple or compound module
\end{itemize}

\tbf{abstract} \textit{(GateDescription)}: base class for entries describing a gate

\begin{itemize}
  \item \tbf{n} (\textit{name}, string): gate name
  \item \tbf{i} (\textit{index}, int): gate index if vector, -1 otherwise
  \item \tbf{o} (\textit{isOutput}, bool): whether the gate is input or output
\end{itemize}

\tbf{abstract} \textit{(ConnectionDescription)}: base class for entries describing a connection

\begin{itemize}
  \item \tbf{dm} (\textit{destModuleId}, int): id of the destination module
  \item \tbf{dg} (\textit{destGateId}, int): id of the gate at the destination module
\end{itemize}

\tbf{abstract} \textit{(MessageDescription)}: base class for entries describing a message

\begin{itemize}
  \item \tbf{tid} (\textit{messageTreeId}, msgid\_t): id of the message inherited by dup
  \item \tbf{eid} (\textit{messageEncapsulationId}, msgid\_t): id of the message inherited by encapsulation
  \item \tbf{etid} (\textit{messageEncapsulationTreeId}, msgid\_t): id of the message inherited by both dup and encapsulation
  \item \tbf{c} (\textit{messageClassName}, string): C++ class name of the message
  \item \tbf{n} (\textit{messageName}, string): message name
  \item \tbf{k} (\textit{messageKind}, short): message kind
  \item \tbf{p} (\textit{messagePriority}, short): message priority
  \item \tbf{l} (\textit{messageLength}, int64\_t): message length in bits
  \item \tbf{er} (\textit{hasBitError}, bool): true indicates that the message has bit errors
  \item \tbf{m} (\textit{ownerModuleId}, int): id of the owner module where the message was found or -1 (FES)
  \item \tbf{sm} (\textit{senderModuleId}, int): id of the source module where the message was sent from
  \item \tbf{sg} (\textit{senderGateId}, int): id of the gate at the source module from which the message is being sent
  \item \tbf{st} (\textit{sendingTime}, simtime\_t): simulation time when the message was sent
  \item \tbf{am} (\textit{arrivalModuleId}, int): id of the destination module where the message was sent to
  \item \tbf{ag} (\textit{arrivalGateId}, int): id of the gate at the source module from which the message is being sent
  \item \tbf{at} (\textit{arrivalTime}, simtime\_t): simulation time when the message will arrive
  \item \tbf{d} (\textit{detail}, string): detailed information of message content when recording message data is turned on
  \item \tbf{pe} (\textit{previousEventNumber}, eventnumber\_t): event number from which the message being cloned was sent, or -1 if the message was sent from initialize
\end{itemize}

\tbf{abstract} \textit{(ModuleDisplayString)}: base class for entries describing a module display string

\begin{itemize}
  \item \tbf{d} (\textit{displayString}, string): the new display string
\end{itemize}

\tbf{abstract} \textit{(GateDisplayString)}: base class for entries describing a gate display string

\begin{itemize}
  \item \tbf{d} (\textit{displayString}, string): the new display string
\end{itemize}

\tbf{abstract} \textit{(ConnectionDisplayString)}: base class for entries describing a connection display string

\begin{itemize}
  \item \tbf{d} (\textit{displayString}, string): the new display string
\end{itemize}

\tbf{abstract} \textit{(MessageDisplayString)}: base class for entries describing a message display string

\begin{itemize}
  \item \tbf{d} (\textit{displayString}, string): the new display string
\end{itemize}

\tbf{CMB} \textit{(ComponentMethodBegin)}: beginning of a call to another module

\begin{itemize}
  \item \tbf{sm} (\textit{sourceComponentId}, int): id of the caller component
  \item \tbf{tm} (\textit{targetComponentId}, int): id of the component being called
  \item \tbf{m} (\textit{methodName}, string): C++ method name
\end{itemize}

\tbf{CME} \textit{(ComponentMethodEnd)}: end of a call to another component

\begin{itemize}
  \item no parameters
\end{itemize}

\tbf{MC} \textit{(ModuleCreated)}: creating a module

\begin{itemize}
  \item no parameters
\end{itemize}

\tbf{MD} \textit{(ModuleDeleted)}: deleting a module

\begin{itemize}
  \item no parameters
\end{itemize}

\tbf{GC} \textit{(GateCreated)}: creating a gate

\begin{itemize}
  \item no parameters
\end{itemize}

\tbf{GD} \textit{(GateDeleted)}: deleting a gate

\begin{itemize}
  \item no parameters
\end{itemize}

\tbf{CC} \textit{(ConnectionCreated)}: creating a connection

\begin{itemize}
  \item no parameters
\end{itemize}

\tbf{CD} \textit{(ConnectionDeleted)}: deleting a connection

\begin{itemize}
  \item no parameters
\end{itemize}

\tbf{MDC} \textit{(ModuleDisplayStringChanged)}: a module display string change

\begin{itemize}
  \item no parameters
\end{itemize}

\tbf{GDC} \textit{(GateDisplayStringChanged)}: a gate display string change

\begin{itemize}
  \item no parameters
\end{itemize}

\tbf{CDC} \textit{(ConnectionDisplayStringChanged)}: a connection display string change

\begin{itemize}
  \item no parameters
\end{itemize}

\tbf{EDC} \textit{(MessageDisplayStringChanged)}: a message display string change

\begin{itemize}
  \item no parameters
\end{itemize}

\tbf{CM} \textit{(CreateMessage)}: creating a message

\begin{itemize}
  \item no parameters
\end{itemize}

\tbf{CL} \textit{(CloneMessage)}: cloning a message either via the copy constructor or by dup

\begin{itemize}
  \item \tbf{cid} (\textit{cloneId}, msgid\_t): id of the original message that is cloned
\end{itemize}

\tbf{DM} \textit{(DeleteMessage)}: deleting a message

\begin{itemize}
  \item no parameters
\end{itemize}

\tbf{EN} \textit{(EncapsulatePacket)}: encapsulating a packet

\begin{itemize}
  \item \tbf{cid} (\textit{encapsulatedPacketId}, msgid\_t): id of the encapsulated packet
\end{itemize}

\tbf{DE} \textit{(DecapsulatePacket)}: decapsulating a packet

\begin{itemize}
  \item \tbf{cid} (\textit{encapsulatedPacketId}, msgid\_t): id of the encapsulated packet
\end{itemize}

\tbf{BS} \textit{(BeginSend)}: beginning to send a message

\begin{itemize}
  \item \tbf{sd} (\textit{sendDelay}, simtime\_t): send after this delay
  \item \tbf{up} (\textit{isUpdate}, bool): whether this is a transmission update
  \item \tbf{tx} (\textit{transmissionId}, txid\_t): for pairing transmission updates with the original transmission
\end{itemize}

\tbf{ES} \textit{(EndSend)}: prediction of the arrival of a message, only a message reference because can't be alone

\begin{itemize}
  \item \tbf{i} (\textit{isDeliveredImmediately}, bool): true indicates the message is delivered
\end{itemize}

\tbf{SD} \textit{(SendDirect)}: sending a message directly to a destination gate

\begin{itemize}
  \item \tbf{sm} (\textit{senderModuleId}, int): id of the source module from which the message is being sent
  \item \tbf{dm} (\textit{destModuleId}, int): id of the destination module to which the message is being sent
  \item \tbf{dg} (\textit{destGateId}, int): id of the gate at the destination module to which the message is being sent
  \item \tbf{pd} (\textit{propagationDelay}, simtime\_t): propagation delay as the message is propagated through the connection
  \item \tbf{td} (\textit{transmissionDelay}, simtime\_t): transmission duration as the whole message is sent from the source gate
  \item \tbf{rd} (\textit{remainingDuration}, simtime\_t): remaining transmission time (if packet is a tx update)
\end{itemize}

\tbf{SH} \textit{(SendHop)}: sending a message through a connection identified by its source module and gate id

\begin{itemize}
  \item \tbf{sm} (\textit{senderModuleId}, int): id of the source module from which the message is being sent
  \item \tbf{sg} (\textit{senderGateId}, int): id of the gate at the source module from which the message is being sent
  \item \tbf{pd} (\textit{propagationDelay}, simtime\_t): propagation delay as the message is propagated through the connection
  \item \tbf{td} (\textit{transmissionDelay}, simtime\_t): transmission duration as the whole message is sent from the source gate
  \item \tbf{rd} (\textit{remainingDuration}, simtime\_t): remaining transmission time (if packet is a tx update)
  \item \tbf{d} (\textit{discard}, bool): whether the channel has discarded the message
\end{itemize}

\tbf{CE} \textit{(CancelEvent)}: canceling an event caused by sending a self message

\begin{itemize}
  \item no parameters
\end{itemize}

\tbf{MF} \textit{(ModuleFound)}: a module found in the simulation while traversing the modules (used in snapshots)

\begin{itemize}
  \item \tbf{\#} (\textit{lastSeenEventNumber}, eventnumber\_t): event number of the module created entry
  \item \tbf{ei} (\textit{lastSeenEntryIndex}, int): index of the module created entry
\end{itemize}

\tbf{GF} \textit{(GateFound)}: a gate found in the simulation while traversing the modules (used in snapshots)

\begin{itemize}
  \item \tbf{\#} (\textit{lastSeenEventNumber}, eventnumber\_t): event number of the gate created entry
  \item \tbf{ei} (\textit{lastSeenEntryIndex}, int): index of the gate created entry
\end{itemize}

\tbf{CF} \textit{(ConnectionFound)}: a connection found in the simulation while traversing the modules (used in snapshots)

\begin{itemize}
  \item \tbf{\#} (\textit{lastSeenEventNumber}, eventnumber\_t): event number of the connection created entry
  \item \tbf{ei} (\textit{lastSeenEntryIndex}, int): index of the connection created entry
\end{itemize}

\tbf{EF} \textit{(MessageFound)}: a message found in the future event queue (FES) or while traversing the modules (used in snapshots)

\begin{itemize}
  \item no parameters
\end{itemize}

\tbf{MDF} \textit{(ModuleDisplayStringFound)}: a module display string found (used in snapshots)

\begin{itemize}
  \item \tbf{\#} (\textit{lastSeenEventNumber}, eventnumber\_t): event number of the module display string changed entry
  \item \tbf{ei} (\textit{lastSeenEntryIndex}, int): index of the module display string changed entry
\end{itemize}

\tbf{GDF} \textit{(GateDisplayStringFound)}: a gate display string found (used in snapshots)

\begin{itemize}
  \item \tbf{\#} (\textit{lastSeenEventNumber}, eventnumber\_t): event number of the gate display string changed entry
  \item \tbf{ei} (\textit{lastSeenEntryIndex}, int): index of the gate display string changed entry
\end{itemize}

\tbf{CDF} \textit{(ConnectionDisplayStringFound)}: a connection display string found (used in snapshots)

\begin{itemize}
  \item \tbf{\#} (\textit{lastSeenEventNumber}, eventnumber\_t): event number of the connection display string changed entry
  \item \tbf{ei} (\textit{lastSeenEntryIndex}, int): index of the connection display string changed entry
\end{itemize}

\tbf{EDF} \textit{(MessageDisplayStringFound)}: a message display string found (used in snapshots)

\begin{itemize}
  \item \tbf{\#} (\textit{lastSeenEventNumber}, eventnumber\_t): event number of the message display string changed entry
  \item \tbf{ei} (\textit{lastSeenEntryIndex}, int): index of the message display string changed entry
\end{itemize}

\tbf{BU} \textit{(Bubble)}: display a bubble message

\begin{itemize}
  \item \tbf{id} (\textit{moduleId}, int): id of the module which printed the bubble message
  \item \tbf{txt} (\textit{text}, string): displayed message text
\end{itemize}

\tbf{abstract} \textit{(CustomReference)}: custom data reference provided by OMNeT users

\begin{itemize}
  \item \tbf{t} (\textit{type}, string): user specified unique type
  \item \tbf{k} (\textit{key}, long): unique key that links custom entries together
\end{itemize}

\tbf{abstract} \textit{(CustomDescription)}: custom data provided by OMNeT users

\begin{itemize}
  \item \tbf{c} (\textit{content}, string): user specified content (text, CSV, XML, JSON, etc.)
\end{itemize}

\tbf{CUC} \textit{(CustomCreated)}: created a custom data object

\begin{itemize}
  \item no parameters
\end{itemize}

\tbf{CUD} \textit{(CustomDeleted)}: deleted a custom data object

\begin{itemize}
  \item no parameters
\end{itemize}

\tbf{CUM} \textit{(CustomChanged)}: changed a custom data object

\begin{itemize}
  \item no parameters
\end{itemize}

\tbf{CUF} \textit{(CustomFound)}: found a custom data object (used in snapshots)

\begin{itemize}
  \item \tbf{\#} (\textit{lastSeenEventNumber}, eventnumber\_t): event number of the corresponding created entry
  \item \tbf{ei} (\textit{lastSeenEntryIndex}, int): index of the corresponding created begin entry
\end{itemize}

\tbf{CU} \textit{(Custom)}: custom data provided by OMNeT users

\begin{itemize}
  \item no parameters
\end{itemize}
% @ENDFILE

%%% Local Variables:
%%% mode: latex
%%% TeX-master: "usman"
%%% End:
