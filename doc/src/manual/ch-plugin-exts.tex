\chapter{Customizing and Extending {\opp}}
\label{cha:plugin-exts}

\section{Overview}
\label{sec:plugin-exts:overview}

{\opp} is an open system, and several details of its operation can be
customized and extended by writing C++ code. Some extension interfaces
have already been covered in other chapters:

\begin{itemize}
   \item \textit{Defining new NED functions} was described in
     \ref{sec:sim-lib:defining-ned-functions}
   \item \textit{Defining new result filters and recorders} was described
     in \ref{sec:simple-modules:writing-result-filters}
\end{itemize}

This chapter will begin by introducing some infrastructure features that
are useful for extensions:

\begin{itemize}
   \item \textit{Config options}. This facility lets other extensions classes
     define their own configuration options.
   \item \textit{Simulation lifecycle listeners} allow extensions to get
     notified when a network is set up, simulation is started, paused or resumed,
     the simulation ended successfully or with an error, and so on.
   \item \cclass{cEvent} lets extensions schedule actions for certain simulation
     times. This is especially useful for custom event schedulers that we'll
     cover later in this chapter.
\end{itemize}

Then we will continue with the descriptions of the following extension
interfaces:

\begin{itemize}
   \item \cclass{cRNG} lets one add new random number generator algorithms.
   \item \cclass{cScheduler} is an interface for event schedulers.
     This extension interface allows for implementing real-time,
     hardware-in-the-loop, distributed and distributed parallel simulation.
   \item \cclass{cFutureEventSet}. This extension interface allows one to
     replace the data structure used for storing future events during
     simulation, i.e. the FES. This may make sense for specialized
     workloads.
   \item \cclass{cFingerprintCalculator}. This extension interface allows one
     to replace or extend the fingerprint computation algorithm.
   \item \cclass{cIOutputScalarManager}. This extension interface allows
     one to create additional means of saving scalar results, for example
     database or CSV output.
   \item \cclass{cIOutputVectorManager}. This extension interface allows
     one to create additional means of saving vector results, for example
     database or CSV output.
   \item \cclass{cIEventlogManager}. This extension interface allows one
     to customize event log recording.
   \item \cclass{cISnapshotManager}. It provides an output stream to which
     snapshots are written.
   \item \cclass{cConfigurationEx}. Configuration provider extension.
     This extension interface lets one replace \ffilename{omnetpp.ini}
     with some other implementation, for example a database.
   \item \textit{User interfaces}. When existing runtime user interfaces
     (Cmdenv, Qtenv) don't suffice, one can create a new one,
     reusing the infrastructure provided by the common base of the three.
\end{itemize}

Many extension interfaces follow a common pattern: one needs to implement a
given interface class (e.g. \cclass{cRNG} for random number generators),
let {\opp} know about it by registering the class with the
\fmac{Register\_Class()} macro, and finally activate it by the appropriate
configuration option (e.g. \ttt{rng-class=MyRNG}). The interface classes
(\cclass{cRNG}, \cclass{cScheduler}, etc.) are documented in the API
Reference.

\begin{note}
A common error is that {\opp} cannot find the class at runtime. When that
happens, make sure the executable actually contains the code of the class.
When linking with a library, over-optimizing linkers (esp. on Unix) tend to
leave out code which seems to be unreferenced by other parts of the
program.
\end{note}

The following sections elaborate on the various extension interfaces.


\section{Adding a New Configuration Option}
\label{sec:plugin-exts:configoption}

\subsection{Registration}
\label{sec:plugin-exts:configoption:registration}

New configuration options need to be declared with one of the appropriate
registration macros. These macros are:

\begin{cpp}
Register_GlobalConfigOption(ID, NAME, TYPE, DEFAULTVALUE, DESCRIPTION)
Register_PerRunConfigOption(ID, NAME, TYPE, DEFAULTVALUE, DESCRIPTION)
Register_GlobalConfigOptionU(ID, NAME, UNIT, DEFAULTVALUE, DESCRIPTION)
Register_PerRunConfigOptionU(ID, NAME, UNIT, DEFAULTVALUE, DESCRIPTION)
Register_PerObjectConfigOption(ID, NAME, KIND, TYPE, DEFAULTVALUE, DESCRIPTION)
Register_PerObjectConfigOptionU(ID, NAME, KIND, UNIT, DEFAULTVALUE, DESCRIPTION)
\end{cpp}

Config options come in three flavors, as indicated by the macro names:

\begin{itemize}
  \item \textit{Global} options affect all configurations (i.e. they are
      only accepted in the \ttt{[General]} section but not in
      \ttt{[Config <name>]} sections)
  \item \textit{Per-Run} options can be specified in any section
      (i.e. both in \ttt{[General]} and in \ttt{[Config <name>]} sections).
      They affect the configuration they occur in.
  \item \textit{Per-Object} options can be specified in any section
      (i.e. both in \ttt{[General]} and in \ttt{[Config <name>]} sections).
      They are specific to an object or group of objects. Their names
      must always contain a hyphen (-) character so that they can be
      distinguished from module/channel parameter assignments when they
      occur in ini files.
\end{itemize}

The macro arguments are as follows:
\begin{itemize}
  \item \textit{ID} is a C++ identifier that becomes the name of a global
        variable, a pointer to a \cclass{cConfigOption} object that the
        macro creates. It lets you refer to the configuration option,
        e.g. when querying its value using \cclass{cConfiguration}'s member
        functions.
  \item \textit{NAME} is the name of the option (a string).
  \item \textit{KIND} applies to per-object configuration options, and
        clarifies what kind of objects the option applies to. Its value
        must be one of: \ttt{KIND\_COMPONENT} (module or channel),
        \ttt{KIND\_CHANNEL}, \ttt{KIND\_MODULE} (simple or compound module),
        \ttt{KIND\_SIMPLE\_MODULE}, \ttt{KIND\_PARAMETER} (module or channel
        parameter), \ttt{KIND\_STATISTIC} (statistic declared in NED
        via \fprop{@statistic}), \ttt{KIND\_SCALAR} (output scalar),
        \ttt{KIND\_VECTOR} (output vector), \ttt{KIND\_UNSPECIFIED\_TYPE}
        (only used for the \fconfig{typename} option), \ttt{KIND\_OTHER}
        (anything else).
  \item \textit{TYPE} is the data type of the config option; it must be one of:
        \ttt{CFG\_BOOL}, \ttt{CFG\_INT}, \ttt{CFG\_DOUBLE}, \ttt{CFG\_STRING},
        \ttt{CFG\_FILENAME}, \ttt{CFG\_FILENAMES}, \ttt{CFG\_PATH}, \ttt{CFG\_CUSTOM}.
        The most significant difference between filesystem-related types
        (filename, filenames, path) and plain strings is that relative
        filenames and paths are automatically converted to absolute
        when the configuration is read, with the base directory being the
        location of the ini file where the configuration entry was read from.
  \item \textit{UNIT} is a string that names the measurement unit in which
        the option's value is to be interpreted; it implies type \ttt{CFG\_DOUBLE}.
  \item \textit{DEFAULTVALUE} is the default value in textual form (string);
        this should be \ttt{nullptr} if the option has no default value.
  \item \textit{DESCRIPTION} is an arbitrarily long string that describes
        the purpose and and operation of the option. It will be used in
        help texts etc.
\end{itemize}

For example, the \ttt{debug-on-errors} option is declared in the following way:

\begin{cpp}
Register_GlobalConfigOption(CFGID_DEBUG_ON_ERRORS, "debug-on-errors",
    CFG_BOOL, "false", "When enabled, runtime errors will cause...");
\end{cpp}

The macro will register the option, and also declare the
\ttt{CFGID\_DEBUG\_ON\_ERRORS} variable as pointer to a
\cclass{cConfigOption}. The variable can be used later as a ``handle'' when
reading the option value from the configuration database.

\subsection{Reading the Value}
\label{sec:plugin-exts:configoption:reading-values}

The configuration is accessible via the \ffunc{getConfig()} method of \cclass{cEnvir}.
It returns a pointer to the configuration object (\cclass{cConfiguration}):

\begin{cpp}
cConfiguration *config = getEnvir()->getConfig();
\end{cpp}

\cclass{cConfiguration} provides several methods for querying the configuration.

\begin{note}
The configuration object provides a flattened view of the ini file. Sections
inheriting from each other are merged. Configuration options provided
on the command line in the form \ttt{--option=value} are added first to the object.
This ensures that the command line options take precedence over the values specified
in the INI file.
\end{note}

\begin{cpp}
const char *getAsCustom(cConfigOption *entry, const char *fallbackValue=nullptr);
bool getAsBool(cConfigOption *entry, bool fallbackValue=false);
long getAsInt(cConfigOption *entry, long fallbackValue=0);
double getAsDouble(cConfigOption *entry, double fallbackValue=0);
std::string getAsString(cConfigOption *entry, const char *fallbackValue="");
std::string getAsFilename(cConfigOption *entry);
std::vector<std::string> getAsFilenames(cConfigOption *entry);
std::string getAsPath(cConfigOption *entry);
\end{cpp}

\textit{fallbackValue} is returned if the value is not specified in the configuration,
and there is no default value.

\begin{cpp}
bool debug = getEnvir()->getConfig()->getAsBool(CFGID_PARSIM_DEBUG);
\end{cpp}


\section{Simulation Lifetime Listeners}
\label{sec:plugin-exts:simulation-lifetime-listeners}

\cclass{cISimulationLifecycleListener} is a callback interface for
receiving notifications at various stages of simulations: setting up,
running, tearing down, etc. Extension classes such as custom event
schedulers often need this functionality for performing initalization
and various other tasks.

Listeners of the type \cclass{cISimulationLifecycleListener} need to be
added to \cclass{cEnvir} with its \ffunc{addLifecycleListener()} method,
and removed with \ffunc{removeLifecycleListener()}.

\begin{cpp}
cISimulationLifecycleListener *listener = ...;
getEnvir()->addLifecycleListener(listener);
// and finally:
getEnvir()->removeLifecycleListener(listener);
\end{cpp}

To implement a simulation lifecycle listener, subclass from
\cclass{cISimulationLifecycleListener}, and override its
\ffunc{lifecycleEvent()} method. It has the following signature:

\begin{cpp}
virtual void lifecycleEvent(SimulationLifecycleEventType eventType, cObject *details) = 0;
\end{cpp}

%% TODO names too long? cISimulationLifecycleListener, SimulationLifecycleEventType... remove "Simulation"?

Event type is one of the following. Their names are fairly self-describing,
but the API documentation contains more precise information.

\begin{itemize}[noitemsep]
  \item \ttt{LF\_ON\_STARTUP}
  \item \ttt{LF\_PRE\_NETWORK\_SETUP}, \ttt{LF\_POST\_NETWORK\_SETUP}
  \item \ttt{LF\_PRE\_NETWORK\_INITIALIZE}, \ttt{LF\_POST\_NETWORK\_INITIALIZE}
  \item \ttt{LF\_ON\_SIMULATION\_START}
  \item \ttt{LF\_ON\_SIMULATION\_PAUSE}, \ttt{LF\_ON\_SIMULATION\_RESUME}
  \item \ttt{LF\_ON\_SIMULATION\_SUCCESS}, \ttt{LF\_ON\_SIMULATION\_ERROR}
  \item \ttt{LF\_PRE\_NETWORK\_FINISH}, \ttt{LF\_POST\_NETWORK\_FINISH}
  \item \ttt{LF\_ON\_RUN\_END}
  \item \ttt{LF\_PRE\_NETWORK\_DELETE}, \ttt{LF\_POST\_NETWORK\_DELETE}
  \item \ttt{LF\_ON\_SHUTDOWN}
\end{itemize}

The \textit{details} argument is currently \ttt{nullptr}; further {\opp}
versions may pass extra information in it. Notifications always refer to
the active simulation in case there're more (see \cclass{cSimulation}'s
\ffunc{getActiveSimulation()}).

Simulation lifecycle listeners are mainly intended for use by classes that
extend the simulator's functionality, for example custom event schedulers
and output vector/scalar managers. The lifecycle of such an extension
object is managed by {\opp}, so one can use their constructor to create and
add the listener object to \cclass{cEnvir}, and the destructor to remove
and delete it. The code is further simplified if the extension object
itself implements \cclass{cISimulationLifecycleListener}:

%% * For standalone listeners that have no other way to be invoked, one can
% * use the <tt>EXECUTE\_ON\_STARTUP(code)</tt> macro to automatically add a%
% * listener when the program starts:
% *
% * <pre>
% * EXECUTE\_ON\_STARTUP(getEnvir()->addLifecycleListener(new MyListener()));
% * </pre>

\begin{cpp}
class CustomScheduler : public cScheduler, public cISimulationLifecycleListener
{
  public:
    CustomScheduler() { getEnvir()->addLifecycleListener(this); }
    ~CustomScheduler() { getEnvir()->removeLifecycleListener(this); }
    //...
};
\end{cpp}


\section{cEvent}
\label{sec:plugin-exts:cevent}

\cclass{cEvent} represents an event in the discrete event simulator. When
events are scheduled, they are inserted into the future events set (FES).
During the simulation, events are removed from the FES and executed one by
one in timestamp order. A \cclass{cEvent} is executed by invoking its
\ffunc{execute()} member function. \ffunc{execute()} should be overridden
in subclasses to carry out the actions associated with the event.

\begin{note}
\cclass{cMessage} is also a subclass of \cclass{cEvent}. Its
\ffunc{execute()} method calls the \ffunc{handleMessage()} method of the
message's destination module, or switches to the coroutine of its
\ffunc{activity()} method.
\end{note}

\ffunc{execute()} has the following signature:

\begin{cpp}
virtual void execute() = 0;
\end{cpp}

Raw (non-message) event objects are an internal mechanism of the {\opp}
simulation kernel, and should not used in programming simulation models.
However, they can be very useful when implementing custom event schedulers.
For example, in co-simulation, events that occur in the other simulator may
be represented with a \cclass{cEvent} in {\opp}. Simulation time limit
is also implemented with a custom \cclass{cEvent}.


\section{Defining a New Random Number Generator}
\label{sec:plugin-exts:rng}

This interface lets one add new RNG implementations (see section
\ref{sec:sim-lib:random-number-generators}) to {\opp}.
The motivation might be achieving integration with external software (for
example something like Akaroa), or exactly replicating the trajectory of a
simulation ported from another simulation framework that uses a different
RNG.

The new RNG C++ class must implement the \cclass{cRNG} interface,
and can be activated with the \fconfig{rng-class} configuration option.


\section{Defining a New Event Scheduler}
\label{sec:plugin-exts:scheduler}

This extension interface lets one replace the event scheduler class with a
custom one, which is the key for implementing many features including
cosimulation, real-time simulation, network or device emulation, and
distributed simulation.

The job of the event scheduler is to always return the next event to be
processed by the simulator. The default implementation returns the first
event in the future events list. Other variants:

\begin{itemize}
\item For real-time simulation, this scheduler is replaced with one augmented
    with \textit{wait} calls (e.g. \texttt{usleep()}) that synchronize the
    simulation time to the system clock. There are several options on what
    should happen if the simulation time has already fallen behind: one may
    re-adjust the reference time, leave it unchanged in the hope of catching up
    later, or stop with an error message.

\item For emulation, the real-time scheduler is augmented with code that captures
    packets from real network devices, and inserts them into the simulation.
    INET Framework, the main protocol simulation package for {\opp}, contains
    an emulation scheduler. It uses the \textit{pcap} library to capture
    packets, and raw sockets to send packets to a real network device.
    Emulation in INET also involves \textit{header serializer} classes that
    convert between protocol headers and their C++ object representations used
    within the simulation.

\item For parallel simulation (see chapter \ref{cha:parallel-exec}), the
    scheduler is modified to listen for messages arriving from other logical
    processes (LPs), and inserts them into the simulation. The scheduler also
    blocks the simulation when it is not safe to execute the next event due to
    potential causality violation, until clearance arrives from other LPs to
    continue in the form of a null message.

\item {\opp} supports distributed simulation using HLA (IEEE 1516) \footnote{The
    source code for the HLA and SystemC integration features are not open
    source, but they are available to researchers on request free of charge.}
    as well. The scheduler plays the role of the HLA Federate Ambassador, is
    responsible for exchanging messages (interactions, change notifications,
    etc.) with other federates, and performs time regulation.

\item {\opp} also supports mixing SystemC (IEEE 1666-2005) modules with {\opp}
    modules in the simulation. When this feature is enabled, there are two
    future event lists in the simulation, {\opp}'s and SystemC's, and a
    special scheduler takes care that events are consumed from both lists in
    increasing timestamp order. This method of performing mixed simulations is
    orders of magnitude faster and also more flexible than letting the two
    simulators execute in separate processes and communicate over a pipe or
    socket connection.
\end{itemize}

The scheduler C++ class must implement the \cclass{cScheduler} interface,
and can be activated with the \fconfig{scheduler-class} configuration option.

Simulation lifetime listeners and the \cclass{cEvent} class can be extremely
useful when implementing certain types of event schedulers.

To see examples of scheduler classes, check the
\cclass{cSequentialScheduler} and \cclass{cRealTimeScheduler} classes in
the simulation kernel, \ttt{cSocketRTScheduler} which is part of the
\textit{Sockets} sample simulation, or \cclass{cParsimSynchronizer} and its
subclasses that are part of the parallel simulation support of {\opp}.


\section{Defining a New FES Data Structure}
\label{sec:plugin-exts:fes}

This extension interface allows one to replace the data structure used for
storing future events during simulation, i.e. the FES. Replacing the FES
may make sense for specialized workloads, or for the purpose of performance
comparison of various FES algorithms. (The default, binary heap based FES
implementation is a good choice for general workloads.)

The FES C++ class must implement the \cclass{cFutureEventSet} interface,
and can be activated with the \fconfig{futureeventset-class} configuration option.


\section{Defining a New Fingerprint Algorithm}
\label{sec:plugin-exts:fingerprint}

This extension interface allows one to replace or extend the fingerprint
computation algorithm (see section \ref{sec:testing:fingerprint-tests}).

% TODO \cclass{cFingerprintCalculator}

The fingerprint computation class must implement the \cclass{cFingerprintCalculator} interface,
and can be activated with the \fconfig{fingerprintcalculator-class} configuration option.


\section{Defining a New Output Scalar Manager}
\label{sec:plugin-exts:outputscalarmanager}

An output scalar manager handles the recording the scalar and histogram
output data. The default output scalar manager is
\cclass{cFileOutputScalarManager} that saves data into \ttt{.sca} files.
This extension interface allows one to create additional means of saving
scalar and histogram results, for example database or CSV output.

The new class must implement \cclass{cIOutputScalarManager}, and can be
activated with the \fconfig{outputscalarmanager-class} configuration
option.


\section{Defining a New Output Vector Manager}
\label{sec:plugin-exts:outputvectormanager}

An output vector manager handles the recording output vectors, produced
for example by \cclass{cOutVector} objects. The default output vector
manager is \cclass{cIndexedFileOutputVectorManager} that saves data into
\ttt{.vec} files, indexed in separate \ttt{.vci} files. This extension
interface allows one to create additional means of saving vector results,
for example database or CSV output.

The new class must implement the \cclass{cIOutputVectorManager} interface,
and can be activated with the \fconfig{outputvectormanager-class}
configuration option.


\section{Defining a New Eventlog Manager}
\label{sec:plugin-exts:eventlogmanager}

An eventlog manager handles the recording of simulation history into an
event log (see \ref{cha:eventlog}). The default eventlog manager is
\cclass{EventlogFileManager}, which records into file, and also allows for
some filtering. By replacing the default eventlog manager class, one can
introduce additional filtering, record into a different file format or
to different storage (e.g. to a database or a remote vizualizer).

The new class must implement the \cclass{cIEventlogManager} interface,
and can be activated with the \fconfig{eventlogmanager-class} configuration
option.


\section{Defining a New Snapshot Manager}
\label{sec:plugin-exts:snapshotmanager}

A snapshot manager provides an output stream to which snapshots are written
(see section \ref{sec:sim-lib:snapshots}). The default snapshot manager
is \cclass{cFileSnapshotManager}.

The new class must implement the \cclass{cISnapshotManager} interface, and
can be activated with the \fconfig{snapshotmanager-class} configuration
option.


\section{Defining a New Configuration Provider}
\label{sec:plugin-exts:configuration-providers}

\subsection{Overview}
\label{sec:plugin-exts:configprovider-overview}

The configuration provider extension lets one replace ini files
with some other storage implementation, for example a database.
The configuration provider C++ class must implement the
\cclass{cConfigurationEx} interface, and can be activated with
the \fconfig{configuration-class} configuration option.

The \cclass{cConfigurationEx} interface abstracts the inifile-based
data model to some degree. It assumes that the configuration data
consists of several \textit{named configurations}. Before every
simulation run, one of the \textit{named configurations} is
activated, and from then on, all queries into the configuration
operate on the \textit{active named configuration} only.

It practice, you will probably use the \cclass{SectionBasedConfiguration}
class (in \ttt{src/envir}) or subclass from it, because it already
implements a lot of functionality that you would otherwise have to.

\cclass{SectionBasedConfiguration} does not assume ini files or
any other particular storage format; instead, it accepts
an object that implements the \cclass{cConfigurationReader}
interface to provide the data in raw form to it.
The default implementation of \cclass{cConfigurationReader} is
\cclass{InifileReader}.

\subsection{The Startup Sequence}
\label{sec:plugin-exts:configprovider-startup-sequence}

From the configuration extension's point of view, the startup sequence
looks like the following (see \ttt{src/envir/startup.cc} in the source code):

\begin{enumerate}
  \item First, ini files specified on the command-line are read into a
     \textit{boot-time configuration object}. The boot-time configuration
     is always a \cclass{SectionBasedConfiguration} with
     \cclass{InifileReader}.
  \item Shared libraries are loaded (see the \fopt{-l} command-line option,
     and the \fconfig{load-libs} configuration option). This allows
     configuration classes to come from shared libraries.
  \item The \fconfig{configuration-class} configuration option is examined.
     If it is present, a configuration object of the given class
     is instantiated, and replaces the boot-time configuration.
     The new configuration object is initialized from the boot-time
     configuration, so that it can read parameters (e.g. database
     connection parameters, XML file name, etc) from it.
     Then the boot-time configuration object is deallocated.
  \item The \fconfig{load-libs} option from the new configuration object is
     processed.
  \item Then everything goes on as normally, using the new configuration
     object.
\end{enumerate}

\subsection{Providing a Custom Configuration Class}
\label{sec:plugin-exts:configprovider:custom-class}

To replace the configuration object with a custom implementation,
one needs to subclass \cclass{cConfigurationEx}, register the new class,

\begin{cpp}
#include "cconfiguration.h"

class CustomConfiguration : public cConfigurationEx
{
   ...
};

Register_Class(CustomConfiguration);
\end{cpp}

and then activate it in the boot-time configuration:

\begin{inifile}
[General]
configuration-class = CustomConfiguration
\end{inifile}


\subsection{Providing a Custom Reader for SectionBasedConfiguration}
\label{sec:plugin-exts:custom-reader-for-sectionbasedconfiguration}

As said already, writing a configuration class from scratch can
be a lot of work, and it may be more practical to reuse
\cclass{SectionBasedConfiguration} with a different configuration
reader class. This can be done with \fconfig{sectionbasedconfig-configreader-class}
config option, which is interpreted by \cclass{SectionBasedConfiguration}.
Specify the following in the boot-time ini file:

\begin{inifile}
[General]
configuration-class = SectionBasedConfiguration
sectionbasedconfig-configreader-class = <new-reader-class>
\end{inifile}

The configuration reader class should look like this:

\begin{cpp}
#include "cconfigreader.h"

class DatabaseConfigurationReader : public cConfigurationReader
{
   ...
};

Register_Class(DatabaseConfigurationReader);
\end{cpp}


\section{Implementing a New User Interface}
\label{sec:plugin-exts:user-interface}

It is possible to extend {\opp} with a new user interface. The new
user interface will have fully equal rights to Cmdenv and Qtenv;
that is, it can be activated by starting the simulation executable with the
\fopt{-u} \textit{<name>} command-line or the \fconfig{user-interface}
configuration option, it can be made the default user interface,
it can define new command-line options and configuration options,
and so on.

User interfaces must implement (i.e. subclass from) \cclass{cRunnableEnvir},
and must be registered to {\opp} with the \fmac{Register\_OmnetApp()} macro.
In practice, you will almost always want to subclass \cclass{EnvirBase}
instead of \cclass{cRunnableEnvir}, because \cclass{EnvirBase} already
implements lots of functionality that otherwise you'd have to.

\begin{note}
If you want something completely different from what \cclass{EnvirBase}
provides, such as embedding the simulation kernel into another
application, then you should be reading section \ref{sec:embedding:embedding},
not this one.
\end{note}

An example user interface:

\begin{cpp}
#include "envirbase.h"

class FooEnv : public EnvirBase
{
    ...
};

Register_OmnetApp("FooEnv", FooEnv, 30, "an experimental user interface");
\end{cpp}

The \ffilename{envirbase.h} header comes from the \ttt{src/envir} directory,
so it is necessary to add it to the include path (\fopt{-I}).

The arguments to \fmac{Register\_OmnetApp()} include the user interface name
(for use with the \fopt{-u} and \fconfig{user-interface} options),
the C++ class that implements it, a weight for default user interface selection
(if \fopt{-u} is missing, the user interface with the largest weight will be
activated), and a description string (for help and other purposes).

The C++ class should implement all methods left pure virtual in
\cclass{EnvirBase}, and possibly others if you want to customize
their behavior. One method that you will surely want to reimplement is
\ffunc{run()} -- this is where your user interface runs. When this method
exits, the simulation program exits.

\begin{note}
A good starting point for implementing your own user interface is
Cmdenv -- just copy and modify its source code to quickly get going.
\end{note}


%%% Local Variables:
%%% mode: latex
%%% TeX-master: "usman"
%%% End:
