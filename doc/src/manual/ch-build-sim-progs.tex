\chapter{Building Simulation Programs}
\label{cha:build-sim-progs}

\section{Overview}
\label{sec:build-sim-progs:overview}

This chapter describes the process and tools for building executable simulation
models from their source code.

As described in the the previous chapters, the source of an {\opp} model usually
contains the following files:

\begin{itemize}
  \item C++ (\ttt{.cc} and \ttt{.h}) files, containing simple module
        implementations and other code;
  \item Message (\ttt{.msg}) files, containing message definitions to be
        translated into C++ classes;
  \item NED (\ttt{.ned}) files with component declarations and topology
        descriptions;
  \item Configuration (\ttt{.ini}) files with model parameter assignments and 
        other settings.
\end{itemize}

The process to turn the source into an executable form is this, in nutshell:

\begin{enumerate}
  \item Message files are translated into C++ using the message compiler,
        \fprog{opp\_msgc}
  \item C++ sources are compiled into object form (\ttt{.o} files)
  \item Object files are linked with the simulation kernel and other 
        libraries to get an executable or a shared library
\end{enumerate}

Note that apart from the first step, the process is the same as building any 
C/C++ program. Also note that NED and ini files do not play a part in this
process, as they are loaded by the simulation program at runtime.

One needs to link with the following libraries:

\begin{itemize}
  \item The simulation kernel and class library\index{simulation!kernel} (the
        \textit{oppsim} library) and its dependencies (\textit{oppenvir},
        \textit{oppcommon}, \textit{oppnedxml}, etc).
  \item Optionally, with one or more user interface libraries
        (\textit{oppqtenv}, \textit{oppcmdenv}). Note that these libraries
        themselves may depend on other libraries.
\end{itemize}

The exact files names of libraries depend on the platform and a number of
additional factors.\footnote{On Unix-like platforms, file names are
prefixed with \ttt{lib}. For debug versions, a \ttt{d} is appended to the
name. Static libraries have the \ttt{.a} suffix (except on Windows where
the file extension is \ttt{.lib}). Shared libraries end in \ttt{.so} on
Unix-like platforms (but \ttt{.dylib} on OS X), and \ttt{.dll} on Windows.}

Figure \ref{fig:ch-build:workflow} shows an overview of the process of
building (and running) simulation programs.

\begin{figure}[htbp]
  \begin{center}
    \includesvg{figures/build-workflow}
    \caption{Building and running simulation}
    \label{fig:ch-build:workflow}
  \end{center}
\end{figure}

You can see that the build process is not complicated. Tools such as
\fprog{make} and \fprog{opp\_makemake}, to be described in the rest of the
chapter, are primarily needed to optimize rebuilds (if a message file has been
translated already, there is no need to repeat the translation for every build
unless the file has changed), and for automation.
 

\section{Using opp\_makemake and Makefiles}
\label{sec:build-sim-progs:opp-makemake}

There are several tools available for managing the build of C/C++ programs.
{\opp} uses the traditional way, Makefiles. Writing a Makefile is usually a
tedious task. However, {\opp} provides a tool that can generate the
Makefile for the user, saving manual labour.

\fprog{opp\_makemake} can automatically generate a Makefile for simulation
programs, based on the source files in the current directory and (optionally)
in subdirectories.

\subsection{Command-line Options}
\label{sec:build-sim-progs:makemake-options}

The most important options accepted by \fprog{opp\_makemake} are:

\begin{itemize}
    \item \fopt{-f}, \fopt{-{}-force} : Force overwriting existing Makefile
    \item \fopt{-o <filename>} : Name of simulation executable or library to be built.
    \item \fopt{-O <directory>}, \fopt{-{}-out <directory>} :
                          Specifies the name of the output directory tree for out-of-directory build
    \item \fopt{-{}-deep} : Generates a "deep" Makefile. A deep Makefile will
                          cover the whole source tree under the make directory,
                          not just files in that directory.
    \item \fopt{-r}, \fopt{-{}-recurse} : Causes make to recursively descend into all subdirectories;
                          subdirectories are expected to contain Makefiles themselves.
    \item \fopt{-X <directory>}, \fopt{-X<directory>}, \fopt{-{}-except <directory>} :
                          With \fopt{-r} or \fopt{-{}-deep}: ignore the given directory.
    \item \fopt{-d<subdir>}, \fopt{-d <subdir>}, \fopt{-{}-subdir <subdir>} :
                          Causes make to switch to the given directory and invoke
                          a Makefile in that directory.
    \item \fopt{-n}, \fopt{-{}-nolink} : Produce object files but do not create an executable or library.
    \item \fopt{-s}, \fopt{-{}-make-so} : Build shared library (\ttt{.so}, \ttt{.dll} or \ttt{.dylib}).
    \item \fopt{-a}, \fopt{-{}-make-lib} : Create static library (\ttt{.a} or \ttt{.lib}).
    \item \fopt{-I<dir>} : Add the given directory to the C++ include path.
    \item \fopt{-D<name>[=<value>]}, \fopt{-D <name>[=v<alue>]}, \fopt{--define <name>[=<value>]} : 
                          Define the given symbol in the C++ compiler.
    \item \fopt{-L<dir>} : Add the given directory to the library path.
    \item \fopt{-l<library>} : Additional library to link against.
\end{itemize}

There are several other options; run \ttt{opp\_makemake -h} to see the complete
list.


\subsection{Basic Use}
\label{sec:build-sim-progs:makemake-basic-use}

Assuming the source files (\ttt{*.ned}, \ttt{*.msg}, \ttt{*.cc}, \ttt{*.h})
are located in a single directory, one can change to that directory and type:

\begin{commandline}
$ opp_makemake
\end{commandline}

This will create a file named \ttt{Makefile}\index{Makefile}. Now, running the
\fprog{make} program will build a simulation executable.

\begin{commandline}
$ make
\end{commandline}

\begin{important}
The generated \ttt{Makefile} will contain the names of the sources files,
so you need to re-run \fprog{opp\_makemake} every time new files are added to 
or removed from the project. 
\end{important}
 
To regenerate an existing \ttt{Makefile}, add the \fopt{-f} option to the
command line, otherwise \fprog{opp\_makemake} will refuse overwriting it.

\begin{commandline}
$ opp_makemake -f
\end{commandline}

The name of the output file\index{output!file} will be derived from
the name of the project directory (see later). It can be overridden
with the \fopt{-o} option:

\begin{commandline}
$ opp_makemake -f -o aloha
\end{commandline}

The generated \ttt{Makefile} supports the following targets:

\begin{itemize}
  \item \ttt{all} : Builds the simulation; this is also the default target.
  \item \ttt{clean} : Deletes files that were produced by the make process.
\end{itemize}


\subsection{Debug and Release Builds}
\label{sec:build-sim-progs:debug-and-release-builds}

\fprog{opp\_makemake} generates a Makefile that can create both release and debug builds.
By default it creates release version, but it is easy to override this behavior by
defining the \ttt{MODE} variable on the \fprog{make} command line.

\begin{commandline}
$ make MODE=debug
\end{commandline}

It is also possible to generate a Makefile that defaults to debug builds.
This can be achieved by adding the \fopt{-{}-mode} option to the \fprog{opp\_makemake}
command line.

\begin{commandline}
$ opp_makemake --mode debug
\end{commandline}

\subsection{Debugging the Makefile}
\label{sec:build-sim-progs:debugging-makefile}

\fprog{opp\_makemake} generates a Makefile that prints only minimal information
during the build process (only the name of the compiled file.) To see the full
compiler commands executed by the Makefile, add the \ttt{V=1} parameter to
the \fprog{make} command line.

\begin{commandline}
$ make V=1
\end{commandline}


\subsection{Using External C/C++ Libraries}
\label{sec:build-sim-progs:using-external-libraries}

If the simulation model relies on an external library, the following
\fprog{opp\_makemake} options can be used to make the simulation
link with the library.

\begin{itemize}
  \item Use the \fopt{-I<dir>} option to specify the location of
        the header files. The directory will be added to the compiler's
        include path. This option is not needed if the header files are
        at a standard location, e.g. installed under \ttt{/usr/include}
        on Linux.
  \item Use the \fopt{-L<dir>} to specify the location of the binaries
        (static or shared library files.) Again, this option is not needed
        if the binaries are at a standard place, e.g. under \ttt{/usr/lib}.
  \item Use the \fopt{-l<libname>} to specify the name of the library.
        The name is normally the file name without the \ttt{lib} prefix and
        the file name extension (e.g. \ttt{.a}, \ttt{.so}, \ttt{.dylib}).
\end{itemize}

For example, linking with a hypothetical \textit{Foo} library installed under
\ttt{opt/} might require the following additional \fprog{opp\_makemake} options:
\ttt{-I/opt/foo/include -L/opt/foo/lib -lfoo}.


\subsection{Building Directory Trees}
\label{sec:build-sim-progs:building-directory-trees}

It is possible to build a whole source directory tree with a single Makefile.
A source tree will generate a single output file (executable or library).
A source directory tree will always have a \ttt{Makefile} in its root,
and source files may be placed anywhere in the tree.

To turn on this option, use the \ttt{opp\_makemake -{}-deep} option.
\ttt{opp\_makemake} will collect all \ttt{.cc} and \ttt{.msg} files from
the whole subdirectory tree, and generate a Makefile that covers all.
To exclude a specific directory, use the \fopt{-X exclude/dir/path}
option. (Multiple \fopt{-X} options are accepted.)

An example:

\begin{commandline}
$ opp_makemake -f --deep -X experimental -X obsolete
\end{commandline}

In the C++ code, include statements should contain the location of the file
relative to the Makefile's location.\footnote{
  Support for deep includes (automatically adding each subdirectory
  to the include path so that includes can be written without specifying
  the location of the file) has been dropped in {\opp} version 5.1, due to
  being error-prone in large projects, and having limited usefulness for
  small projects.}
For example, if \ttt{Foo.h} is under \ttt{utils/common/} in the source
tree, it needs to be included as

\begin{cpp}
#include "utils/common/Foo.h"
\end{cpp}


\subsection{Dependency Handling}
\label{sec:build-sim-progs:dependency-handling}

The \fprog{make} program can utilize dependency information in the Makefile
to shorten build times by omitting build steps whose input has not changed
since the last build. Dependency information is automatically created and kept
up-to-date during the build process.

Dependency information is kept in \ttt{.d} files in the output directory.


\subsection{Out-of-Directory Build}
\label{sec:build-sim-progs:out-of-directory-build}

The build system creates object and executable files in a separate directory, called the
\textit{output directory}. By default, the output directory is \ttt{out/<configname>},
where the \ttt{<configname>} part depends on the compiler toolchain and build mode settings.
(For example, the result of a debug build with GCC will be placed in \ttt{out/gcc-debug}.)
The subdirectory tree inside the output directory will mirror the source directory
structure.

\begin{note}
Generated source files (i.e. those created by \fprog{opp\_msgc}) will be placed in the
source tree rather than the output directory.
\end{note}

By default, the \ttt{out} directory is placed in the project root directory. This location
can be changed with \fprog{opp\_makemake}'s \fopt{-O} option.

\begin{commandline}
$ opp_makemake -O ../tmp/obj
\end{commandline}

\begin{note}
The project directory is identified as the first ancestor of the current directory
that contains a \ttt{.project} file.
\end{note}


\subsection{Building Shared and Static Libraries}
\label{sec:build-sim-progs:building-shared-and-static-libraries}

By default the Makefile will create an executable file, but it is also
possible to build shared or static libraries. Shared libraries
are usually a better choice.

Use \fopt{-{}-make-so} to create shared libraries, and \fopt{-{}-make-lib}
to build static libraries. The \fopt{-{}-nolink} option completely omits
the linking step, which is useful for top-level Makefiles that only invoke
other Makefiles, or when custom linking commands are needed.

\ifcommercial
\begin{note}
The Microsoft Visual C++ compiler handles shared library (DLL) linking
differently, and requires additional steps to compile and link correctly.
Namely, one must choose a \ttt{<symbol>} for the project, and annotate all classes,
functions and variables to be exported from the DLL with \ttt{<symbol>\_API}.
Then, Makefiles need to be generated with the \fopt{-p <symbol>} option.
This option will cause the \ttt{<symbol>\_EXPORT} macro will be passed to
the compiler, causing \ttt{<symbol>\_API} to be defined as \ttt{\_\_declspec(dllexport)} in
the sources. This step allows Visual C++ to correctly generate DLL exports in Windows.
\end{note}

%% TODO meg amit pluszban be kell tenni a headerbe!!!

\fi


\subsection{Recursive Builds}
\label{sec:build-sim-progs:recursive-builds}


The \fopt{-{}-recurse} option enables recursive make; when you build the simulation, make
descends into the subdirectories and runs make in them too.
By default, \fopt{-{}-recurse} decends into all subdirectories; the \fopt{-X <dir>} option
can be used to make it ignore certain subdirectories. This option is especially useful
for top level Makefiles.


The \fopt{-{}-recurse} option automatically discovers subdirectories, but this
is sometimes inconvenient. Your source directory tree may contain
parts which need their own hand written Makefile. This can happen if
you include source files from an other non {\opp} project. With the \fopt{-d <dir>}
or \fopt{-{}-subdir <dir>} option, you can explicitly specify which directories to
recurse into, and also, the directories need not be direct children of the
current directory.


The recursive make options (\fopt{-{}-recurse}, \fopt{-d}, \fopt{-{}-subdir})
imply \fopt{-X}, that is, the directories recursed into will be
automatically excluded from deep Makefiles.


You can control the order of traversal by adding dependencies into
the \ttt{makefrag} file (see \ref{sec:makefrag})

\begin{note}
With \fopt{-d}, it is also possible to create infinite recursions.
\fprog{opp\_makemake} cannot detect them, it is your responsibility that
cycles do not occur.
\end{note}


Motivation for recursive builds:
\begin{itemize}
 \item toplevel Makefile
 \item integrating sources that have their own Makefile
\end{itemize}


\subsection{Customizing the Makefile}
\label{sec:makefrag}

It is possible to add rules or otherwise customize the generated Makefile
by providing a \ffilename{makefrag} file. When you run \fprog{opp\_makemake}, it
will automatically insert the content of the \ffilename{makefrag} file into the
resulting \ffilename{Makefile}. With the \fopt{-i} option, you can also name other
files to be included into the \ttt{Makefile}.

\ffilename{makefrag} will be inserted after the definitions but before the first
rule, so it is possible to override existing definitions and add new
ones, and also to override the default target.

\ttt{makefrag} can be useful if some of your source files are generated
from other files (for example, you use generated NED files), or you need
additional targets in your Makefile or just simply want to override the
default target in the Makefile.

\begin{note}
If you change the content of the \ttt{makefrag} file, you must recreate the
Makefile using the \ttt{opp\_makemake} command.
\end{note}

\subsection{Projects with Multiple Source Trees}
\label{sec:build-sim-progs:projects-with-multiple-source-trees}

In the case of a large project, your source files may be spread across
several directories and your project may generate more than one executable
file (i.e. several shared libraries, examples etc.).

Once you have created your Makefiles with \ttt{opp\_makemake} in
every source directory tree, you will need a toplevel Makefile.
The toplevel Makefile usually calls only the Makefiles
recursively in the source directory trees.


\subsection{A Multi-Directory Example}
\label{sec:build-sim-progs:multi-directory-example}

For a complex example of using \fprog{opp\_makemake}, we will show how to create
the Makefiles for a large project. First, take a look at the
project's directory structure and find the directories that should be used as
source trees:

\begin{verbatim}
project/
    doc/
    images/
    simulations/
    contrib/ <-- source tree (build libmfcontrib.so from this dir)
    core/ <-- source tree (build libmfcore.so from this dir)
    test/ <-- source tree (build testSuite executable from this dir)
\end{verbatim}

Additionally, there are dependencies between these output files: \ttt{mfcontrib}
requires \ttt{mfcore} and \ttt{testSuite} requires \ttt{mfcontrib} (and indirectly
\ttt{mfcore}).

First, we create the Makefile for the core directory. The Makefile will build
a shared lib from all .cc files in the \ttt{core} subtree, and will name it \ttt{mfcore}:

\begin{commandline}
$ cd core && opp_makemake -f --deep --make-so -o mfcore -O out
\end{commandline}

The \ttt{contrib} directory depends on \ttt{mfcore}, so we use the \fopt{-L} and
\fopt{-l} options to specify the library we should link with.

\begin{commandline}
$ cd contrib && opp_makemake -f --deep --make-so -o mfcontrib -O out \
  -I../core -L../out/\$\(CONFIGNAME\)/core -lmfcore
\end{commandline}

The \ttt{testSuite} will be created as an executable file which depends on both
\ttt{mfcontrib} and \ttt{mfcore}.

\begin{commandline}
$ cd test && opp_makemake -f --deep -o testSuite -O out \
    -I../core -I../contrib -L../out/\$\(CONFIGNAME\)/contrib -lmfcontrib
\end{commandline}

Now, let us specify the dependencies among the above directories.
Add the lines below to the \ttt{makefrag} file in the project root directory.

\begin{filelisting}
contrib_dir: core_dir
test_dir: contrib_dir
\end{filelisting}

Now the last step is to create a top-level Makefile in the root of the project that
calls the previously created Makefiles in the correct order. We will use the
\fopt{-{}-nolink} option, exclude every subdirectory from the build (\fopt{-X.}),
and explicitly call the above Makefiles using \fopt{-d <dir>}.
\fprog{opp\_makemake} will automatically include the above created \ttt{makefrag} file.

\begin{commandline}
$ opp_makemake -f --nolink -O out -d test -d core -d contrib -X.
\end{commandline}

\section{Project Features}
\label{sec:build-sim-progs:project-features}

Long compile times are often an inconvenience when working with large
{\opp}-based model frameworks. {\opp} has a facility named \textit{project
features} that lets you reduce build times by excluding or disabling parts
of a large model library. For example, you can disable modules that you do
not use for the current simulation study. The word \textit{feature} refers
to a piece of the project codebase that can be turned off as a whole.

Additional benefits of project features include enforcing cleaner
separation of unrelated parts in the model framework, being able to exclude
code written for other platforms, and a less cluttered model palette in the
NED editor.

\begin{note}
  Modularization could also be achieved via breaking up the model framework
  into several smaller projects, but that would cause other kinds of
  inconveniences for model developers and users alike.
\end{note}

Project features can be enabled/disabled from both the IDE and the command line.
It is possible to query the list of enabled project features, and use this
information in creating a Makefile for the project.


\subsection{What is a Project Feature}
\label{sec:build-sim-progs:project-feature}

Features can be defined per project. As already mentioned, a feature is a piece of the
project codebase that can be turned off as a whole, that is, excluded from the C++ sources
(and thus from the build) and also from NED. Feature definitions are typically written
and distributed by the author of the project; end users are only presented with the
option of enabling/disabling those features. A feature definition contains:

\begin{itemize}
  \item Feature name; for example "UDP" or "Mobility examples".
  \item Feature description; This is a few sentences of text describing what the feature
    is or does; for example "Implementation of the UDP protocol".
  \item Labels; This is a list of labels or keywords that facilitate grouping or finding features.
  \item Initially enabled. This is a boolean flag that determines the initial enablement
    of the feature.
\item Required features; Some features may be built on top of others; for example, a HMIPv6
    protocol implementation relies on MIPv6, which in turn relies on IPv6. Thus, HMIPv6 can
    only be enabled if MIPv6 and IPv6 are enabled as well.
\item NED packages; This is a list of NED package names that identify the code that implements
    the feature. When you disable the feature, NED types defined in those packages and their
    subpackages will be excluded; also, C++ code in the folders that correspond to the packages
    (i.e. in the same folders as excluded NED files) will also be excluded.
\item Extra C++ source folders; If the feature contains C++ code that lives outside NED source
    folders (nontypical), those folders are listed here.
\item Compile options. When the feature is enabled, the compiler options listed
    here are added to the compiler command line of all C++ files in the project.
    Defines (\fopt{-D} options) are treated somewhat specially: the project can
    be set up so that defines go into a generated header file as \ttt{\#define}
    lines instead of being added to the compiler command line. It is
    customary for each feature to have a corresponding symbol
    (\ttt{WITH\_FOO} for a feature called \textit{Foo}), so that other parts of
    the code can contain conditional blocks that are only compiled in when the
    given feature is enabled (or disabled). 
\item Linker options. When the feature is enabled, the linker options listed here are added
    to the linker command line. A typical use of this field is linking with additional
    libraries that the feature's code requires, for example libavcodec.
    Currently only the \fopt{-l} option (\textit{link with library}) is supported here.
\end{itemize}


\subsection{The opp\_featuretool Program}
\label{sec:build-sim-progs:opp-featuretool}

Project features can be queried and manipulated using the \fprog{opp\_featuretool}
program. The first argument to the program must be a command; the most frequently
used ones are \ttt{list}, \ttt{enable} and \ttt{disable}. The operation of commands
can be refined with further options. One can obtain the full list of commands
and options using the \fopt{-h} option.

Here are some examples of using the program.

Listing all features in the project:
\begin{commandline}
$ opp_featuretool list
\end{commandline}

Listing all enabled features in the project:
\begin{commandline}
$ opp_featuretool list -e
\end{commandline}

Enabling all features:
\begin{commandline}
$ opp_featuretool enable all
\end{commandline}

Disabling a specific feature:
\begin{commandline}
$ opp_featuretool disable Foo
\end{commandline}

The following command prints the command line options that should be used
with \ttt{opp\_makemake} to create a Makefile that builds the project with the
currently enabled features:

\begin{commandline}
$ opp_featuretool options
\end{commandline}

The easiest way to pass the output of the above command to \ttt{opp\_makemake}
is the \ttt{\$(\ldots)} shell construct:

\begin{commandline}
$ opp_makemake --deep $(opp_featuretool options)
\end{commandline}

Often it is convenient to put feature defines (e.g. \ttt{WITH\_FOO}) into a
header file instead of passing them to the compiler via \fopt{-D} options.
This makes it easier to detect feature enablements from derived projects,
and also makes it easier for C++ code editors to correctly highlight
conditional code blocks that depend on project features.

The header file can be generated with \ttt{opp\_featuretool} using the
following command:

\begin{commandline}
$ opp_featuretool defines >feature_defines.h
\end{commandline}

At the same time, \fopt{-D} options must be removed from the compiler
command line. \ttt{opp\_featuretool options} has switches to filter them out.
The modified command for Makefile generation:

\begin{commandline}
$ opp_makemake --deep $(opp_featuretool options -fl)
\end{commandline}

It is advisable to create a Makefile rule that regenerates the header file
when feature enablements change:

\begin{filelisting}
feature_defines.h: $(wildcard .oppfeaturestate) .oppfeatures
        opp_featuretool defines >feature_defines.h
\end{filelisting}



\subsection{The .oppfeatures File}
\label{sec:build-sim-progs:oppfeatures-file}

Project features are defined in the \ffilename{.oppfeatures} file in your project's
root directory. This is an XML file, and it has to be written by hand
(there is no specialized editor for it).

The root element is \ttt{<features>}, and it may have several \ttt{<feature>}
child elements, each defining a project feature. The fields of a feature
are represented with XML attributes; attribute names are \ttt{id, name,
description, initiallyEnabled, requires, labels, nedPackages,
extraSourceFolders, compileFlags} and \ttt{linkerFlags}. Items within attributes
that represent lists (\ttt{requires}, \ttt{labels}, etc.) are separated by spaces.

Here is an example feature from the INET Framework:
\begin{xml}
<feature
  id="TCP_common"
  name="TCP Common"
  description = "The common part of TCP implementations"
  initiallyEnabled = "true"
  requires = "IPv4"
  labels = "Transport"
  nedPackages = "inet.transport.tcp_common
                 inet.applications.tcpapp
                 inet.util.headerserializers.tcp"
  extraSourceFolders = ""
  compileFlags = "-DWITH_TCP_COMMON"
  linkerFlags = ""
  />
\end{xml}

Project feature enablements are stored in the \ffilename{.featurestate} file.


\subsection{How to Introduce a Project Feature}
\label{sec:build-sim-progs:introducing-project-features}

If you plan to introduce a project feature in your project, here's what you'll need
to do:

\begin{itemize}
  \item Isolate the code that implements the feature into a separate source directory
        (or several directories). This is because only whole folders can be
        declared as part of a feature, individual source files cannot.

  \item Check the remainder of the project. If you find source lines that reference
        code from the new feature, use conditional compilation (\ttt{\#ifdef WITH\_YOURFEATURE})
        to make sure that code compiles (and either works sensibly or throws an error)
        when the new feature is disabled. (Your feature should define the \ttt{WITH\_YOURFEATURE}
        symbol, i.e. \fopt{-DWITH\_YOURFEATURE} will need to be added to the feature compile flags.)

  \item Add the feature description into the \ttt{.oppfeatures} file of your project.

  \item Test. A rudimentary test is to verify that the project compiles at all,
        both with the new feature enabled and disabled. For projects with many
        features, automated build tests that compile the project using various
        feature configurations can be very useful. Such build tests can be
        written on top of \ttt{opp\_featuretool}.
        
\end{itemize}

%%% Local Variables:
%%% mode: latex
%%% TeX-master: "usman"
%%% End:
